\inputencoding{utf8}
\HeaderA{dde.utils}{Utility Functions for Delay Differential Equation Systems}{dde.utils}
\aliasA{dde.lags}{dde.utils}{dde.lags}
\aliasA{dde.syst}{dde.utils}{dde.syst}
\keyword{method}{dde.utils}
%
\begin{Description}\relax
This is a collection of utility functions called by \code{dde.model} when 
a model defined by delay differential equations is evaluated. None of these 
functions is typically called directly by users.
\end{Description}
%
\begin{Usage}
\begin{verbatim}
  dde.syst(t = NULL,
           y = NULL,
           ic = NULL,
           parms = NULL,
           derparms = NULL,
           delags = NULL,
           codedde = NULL,
           dosing = NULL,
           has.dosing = NULL,
           dose.states = NULL,
           xdata = NULL,
           covdata = NULL,
           scale = NULL,
           check = FALSE)
  dde.lags(parms = NULL,
           derparms = NULL,
           codelags = NULL,
           check = FALSE)
\end{verbatim}
\end{Usage}
%
\begin{Arguments}
\begin{ldescription}
\item[\code{t}] A scalar or a vector of numerical time values.
\item[\code{y}] A vector of system state values.
\item[\code{ic}] A vector of initial conditions, typically returned by 
\code{init.cond}.
\item[\code{parms}] A vector of primary parameters.
\item[\code{derparms}] A list of derived parameters, specified in the \$DERIVED block
of code.
\item[\code{delags}] A vector of delay parameters, typically returned by 
\code{dde.lags}.
\item[\code{codedde}] The content of the R code specified within the \$DDE block in 
the model file.
\item[\code{dosing}] A data.frame of dosing information created by \code{make.dosing}
from instantaneous and zero-order inputs into the system and containing the 
following columns: \begin{description}

\item[TIME] Dosing event times.
\item[CMT] State where the input should be assigned to.
\item[AMT] Amount that should be assigned to system \code{state} at the
corresponding \code{TIME}.
\item[RATE] Rate of input that should be assigned to system \code{CMT} at 
the corresponding \code{TIME}. See 
\code{vignette('scaRabee',package='scaRabee')} for more details about
the interpolation of the input rate at time not specified in
\code{dosing}.
\item[TYPE] An indicator of the type of input. \code{TYPE} is set to 1 if
the record in \code{dosing} correspond an original bolus input; it is 
set to 0 otherwise.

\end{description}


\item[\code{has.dosing}] A logical variable, indicating whether any input has to
be assigned to a system state.
\item[\code{dose.states}] A vector of integers, indicating in which system state one
or more doses have to be assigned to.
\item[\code{xdata}] A vector of times at which the system is being evaluated.
\item[\code{covdata}] A matrix of covariate data extracted from the dataset.
\item[\code{scale}] A vector of system scale, typically returned by 
\code{input.scaling}
\item[\code{check}] An indicator whether checks should be performed to validate 
function inputs
\item[\code{codelags}] The content of the R code specified within the \$LAGS block in
the model file.
\end{ldescription}
\end{Arguments}
%
\begin{Details}\relax
\code{dde.syst} is the function which actually evaluates the system of delay
differential equations specified in the \$DDE block.

\code{dde.lags} is the function which evaluates the code specified in the \$LAG 
block and defines the delays at which the system needs to be computed.

\end{Details}
%
\begin{Author}\relax
Sebastien Bihorel (\email{sb.pmlab@gmail.com})
\end{Author}
%
\begin{SeeAlso}\relax
\code{\LinkA{dde.model}{dde.model}}, \code{\LinkA{make.dosing}{make.dosing}}
\end{SeeAlso}
