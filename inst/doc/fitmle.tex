\inputencoding{utf8}
\HeaderA{fitmle}{Maximum Likelihood Estimator}{fitmle}
\keyword{method}{fitmle}
%
\begin{Description}\relax
\code{fitmle} is a secondary function called during estimation runs. It performs the optimization
of the model parameters by the method of the maximum likelihood, i.e. the minimization of an
objective function defined as twice the log of the exact likelihood of the observed data, given the
structural model, the model of residual variability, and the parameter estimates. This minimization
is performed by the Nelder-Mead simplex algorithm implemented in \code{fminsearch} from the
\pkg{neldermead} package.
\end{Description}
%
\begin{Usage}
\begin{verbatim}
  fitmle(problem = NULL,
         estim.options = NULL,
         files = NULL)
\end{verbatim}
\end{Usage}
%
\begin{Arguments}
\begin{ldescription}
\item[\code{problem}] A list containing the following levels:\begin{description}

\item[data] A list containing the following levels:\begin{description}

\item[xdata] 1 x m matrix of independent variable.
\item[ydata] n x m matrix of observations from model states.
\item[ids] Data.frame of indices for data subsetting (output
from \code{find.id}).

\end{description}

\item[dosing] A list containing the following levels:\begin{description}

\item[history] d x 4 data.frame of dosing history.
\item[ids] data.frame of indices for dosing subsetting
(output from \code{find.id}).

\end{description}

\item[cov] A list containing the following levels:\begin{description}

\item[data] c x t data.frame of covariate history.
\item[ids] Data.frame of indices for cov subsetting (output
from \code{find.id}).

\end{description}

\item[states] Indices of the states to be output by the model.
\item[init] A data.frame of parameter data with the following columns:
'names', 'type', 'value', 'isfix', 'lb', and 'ub'.
\item[debugmode] Logical indicator of debugging mode.
\item[modfun] Model function.
\item[varfun] Variance function; if empty \code{weighting.additive} is
used.
\item[secfun] Secondary parameter function.

\end{description}


\item[\code{estim.options}] A list of estimation options containing two elements
\code{maxiter} (the maximum number of iterations) and \code{maxeval} (the
maximum number of function evaluations).

\item[\code{files}] A list of input used for the analysis. The following elements are
expected and none of them could be null: \begin{description}

\item[data] A .csv file located in the working directory, which contains
the observations of the dependent variable(s) to be modeled. The
expected format of this file is described in details in
\code{vignette('scaRabee',package='scaRabee')}.
\item[param] A .csv file located in the working directory, which contains
the initial guess(es) for the model parameter(s) to be optimized or used
for model simulation. The expected format of this file is described in
details in \code{vignette('scaRabee',package='scaRabee')}.
\item[dose] A .csv file located in the working directory, which contains
the dosing information. The expected format of this file is described in
details in \code{vignette('scaRabee',package='scaRabee')}.
\item[cov] A .csv file located in the working directory, which contains
the values of one or more covariates that may or may or be used within
the model. The expected format of this file is described in details in
\code{vignette('scaRabee',package='scaRabee')}.
\item[model] A .R file located in the 'model.definition'
sub-directory in the working directory, which defines the model. Models
specified with explicit, ordinary or delayed differential equations
should be preferentially defined using the provided templates. More
details about the expected structure of this file is provided in
\code{vignette('scaRabee',package='scaRabee')}, in case the user would
want to develop her/his own template.
\item[var] A .R file located in the 'model.definition' sub-directory
in the working directory, which defines the model of residual
variability. More details about the expected content of this file is
provided in \code{vignette('scaRabee',package='scaRabee')}.
\item[sec] A .R file located in the 'model.definition' sub-directory
in the working directory, which defines the method of computation of
secondary parameters (derived from the fixed or estimated model
parameters). More details about the expected content of this file is
provided in \code{vignette('scaRabee',package='scaRabee')}.

\end{description}


\end{ldescription}
\end{Arguments}
%
\begin{Value}
Return a list with two elements: \code{estimations} which contains the vector
of final parameter estimates and \code{fval} the minimal value of the
objective function.
\end{Value}
%
\begin{Author}\relax
Sebastien Bihorel (\email{sb.pmlab@gmail.com})

Pawel Wiczling
\end{Author}
%
\begin{SeeAlso}\relax
\code{\LinkA{fminsearch}{fminsearch}}
\end{SeeAlso}
