\inputencoding{utf8}
\HeaderA{weighting}{Residual Variability}{weighting}
\keyword{method}{weighting}
%
\begin{Description}\relax
\code{weighting} is a secondary function called during estimation run to
evaluate the model(s) of residual variability specified by the code provided
in the \$VARIANCE block. \code{weighting} is typically not called directly by
users.
\end{Description}
%
\begin{Usage}
\begin{verbatim}
  weighting(parms = NULL,
            derparms = NULL,
            codevar=NULL,
            y=NULL,
            xdata=NULL,
            check=FALSE)
\end{verbatim}
\end{Usage}
%
\begin{Arguments}
\begin{ldescription}
\item[\code{parms}] A vector of primary parameters.
\item[\code{derparms}] A list of derived parameters, specified in the \$DERIVED block
of code.
\item[\code{codevar}] The content of the R code specified within the \$VARIANCE block 
in the model file.
\item[\code{y}] The matrix of structural model predictions.
\item[\code{xdata}] A vector of times at which the system is being evaluated.
\item[\code{check}] An indicator whether checks should be performed to validate 
function inputs
\end{ldescription}
\end{Arguments}
%
\begin{Value}
Return a matrix of numeric values of the same dimension as \code{f}.
\end{Value}
%
\begin{Author}\relax
Sebastien Bihorel (\email{sb.pmlab@gmail.com})
\end{Author}
