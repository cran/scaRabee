\inputencoding{utf8}
\HeaderA{get.events}{Create events from bolus dosing records.}{get.events}
\keyword{method}{get.events}
%
\begin{Description}\relax
\code{get.events} is a secondary function called by \code{dde.model}. It 
creates a data.frame of events from the bolus dosing records found in the 
dataset. \code{get.events} is typically not called directly by users.
\end{Description}
%
\begin{Usage}
\begin{verbatim}
  get.events(bolus = NULL, 
             scale = NULL)
\end{verbatim}
\end{Usage}
%
\begin{Arguments}
\begin{ldescription}
\item[\code{bolus}] b x 4 data.frame providing the instantaneous inputs
\item[\code{scale}] s x 1 vector of scaling factors
\end{ldescription}
\end{Arguments}
%
\begin{Value}
Return a data.frame of events with the following elements: \begin{description}

\item[var] A name of the state affected by the event
\item[time] The time of the event
\item[value] The value associated with the event
\item[method] How the event affects the state ('add' by default)

\end{description}

See \code{\LinkA{events}{events}} for more details
\end{Value}
%
\begin{Author}\relax
Sebastien Bihorel (\email{sb.pmlab@gmail.com})
\end{Author}
%
\begin{SeeAlso}\relax
code\LinkA{events}{events}
\end{SeeAlso}
