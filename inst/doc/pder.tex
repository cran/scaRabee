\inputencoding{utf8}
\HeaderA{pder}{Compute Matrix of Partial Derivatives}{pder}
\keyword{method}{pder}
%
\begin{Description}\relax
\code{pder} is a secondary function called by \code{get.cov.matrix}. It 
computes the matrix of partial derivatives for the model predictions and the
residual variability. \code{pder} is typically not called directly by users.
\end{Description}
%
\begin{Usage}
\begin{verbatim}
  pder(subproblem = NULL,
       x = NULL)
\end{verbatim}
\end{Usage}
%
\begin{Arguments}
\begin{ldescription}
\item[\code{subproblem}] A list containing the following levels:\begin{description}

\item[code] A list of R code extracted from the model file. Depending on 
content of the model file, the levels of this list could be: template,
derived, lags, ode, dde, output, variance, and/or secondary.
\item[method] A character string, indicating the scale of the analysis. Should
be 'population' or 'subject'.
\item[init] A data.frame of parameter data with the following columns:
'names', 'type', 'value', 'isfix', 'lb', and 'ub'.
\item[debugmode] Logical indicator of debugging mode.
\item[modfun] Model function.
\item[hbsize] An integer scalar providing the size of the history buffer for
the solver of delay differential equations. This variable is ignored if
the model file does not contain a \$DDE tag.
\item[data] A list containing the following levels:\begin{description}

\item[xdata] 1 x m matrix of time of observations of the dependent
variables.
\item[data] m x 3 data.frame containing the times of observations of the
dependent variables (extracted from the TIME variable), the indicators 
of the type of dependent variables (extracted from the CMT variable), 
and the actual dependent variable observations (extracted from the DV 
variable).

\end{description}


\item[bolus] bij x 4 data.frame providing the instantaneous inputs for a
treatment and individual.
\item[infusion] fij x (4+c) data.frame providing the zero-order inputs for a
treatment and individual.
\item[cov] mij x c data.frame containing the times of observations of the 
dependent variables (extracted from the TIME variable) and all the
covariates identified for this particular treatment.

\end{description}


\item[\code{x}] The vector of \emph{p} final parameter estimates.
\end{ldescription}
\end{Arguments}
%
\begin{Value}
Return a list containing the \emph{p} x \emph{p} matrices of partial
derivatives for model predictions (\code{mpder}) and residual variability
(\code{wpder}).
\end{Value}
%
\begin{Author}\relax
Sebastien Bihorel (\email{sb.pmlab@gmail.com})
\end{Author}
%
\begin{SeeAlso}\relax
\code{\LinkA{get.cov.matrix}{get.cov.matrix}}
\end{SeeAlso}
