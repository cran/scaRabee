\inputencoding{utf8}
\HeaderA{scarabee.check.reserved}{Check for Reserved Variable Names}{scarabee.check.reserved}
\keyword{method}{scarabee.check.reserved}
%
\begin{Description}\relax
\code{scarabee.check.reserved} is a secondary function called at each 
\pkg{scaRabee} run. It determined whether user-defined parameter names use 
reserved names and if some user-defined parameters are related to the dosing
history. \code{scarabee.check.reserved} is typically not called directly by
users. \code{scarabee.check.reserved} is typically not called directly by
users.
\end{Description}
%
\begin{Usage}
\begin{verbatim}
  scarabee.check.reserved(names = NULL, covnames = NULL)
\end{verbatim}
\end{Usage}
%
\begin{Arguments}
\begin{ldescription}
\item[\code{names}] A vector of parameter names, typically extracted from the file
of parameter definition.
\item[\code{covnames}] A vector of covariate names, typically extracted from the 
data file.
\end{ldescription}
\end{Arguments}
%
\begin{Details}\relax
If one or more user-defined parameters are found to use reserved names, the 
run is stopped and the user is ask to update the name(s) of this(ese) 
parameter(s).

Furthermore, \code{scarabee.check.reserved} determines whether or not the list
of user-defined parameters include D1, D2, ..., D25, or R1, R2, ..., R25. Dn 
and Rn parameters relate to zero-order inputs into the n\textasciicircum{}th state of the system 
and represent the estimated duration and rate of these inputs 
be understood as the duration of any 
\end{Details}
%
\begin{Author}\relax
Sebastien Bihorel (\email{sb.pmlab@gmail.com})
\end{Author}
