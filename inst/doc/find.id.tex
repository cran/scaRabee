\inputencoding{utf8}
\HeaderA{find.id}{Scan for Integers in a Sorted Vector and Return Table of Min and Max Indices}{find.id}
\keyword{method}{find.id}
%
\begin{Description}\relax
\code{find.id} is a secondary function, which main purpose is to split the 
problem into sub-problems. It scans a vector of sorted integers (i.e. the
\code{DoseID} variable from the data, dosing and covariate input files) and
returns a table indicating between indices a particular integer is used.
\end{Description}
%
\begin{Usage}
\begin{verbatim}
  find.id(x)
\end{verbatim}
\end{Usage}
%
\begin{Arguments}
\begin{ldescription}
\item[\code{x}] A vector of sorted integers starting at 1 and increasing to n.
\end{ldescription}
\end{Arguments}
%
\begin{Value}
Returns a data.frame with the following columns: \begin{description}

\item[ID] the unique integers of \code{x}.
\item[starting] first indice of \code{x} where the corresponding integer is
used.
\item[ending] last indice of \code{x} where the corresponding integer is
used.

\end{description}

\end{Value}
%
\begin{Author}\relax
Sebastien Bihorel (\email{sb.pmlab@gmail.com})
\end{Author}
%
\begin{Examples}
\begin{ExampleCode}
  x <- rep(1:3,each=5)
  x
  find.id(x)
\end{ExampleCode}
\end{Examples}
