\inputencoding{utf8}
\HeaderA{bound.parameters}{Forces parameter estimates between defined boundaries.}{bound.parameters}
\keyword{method}{bound.parameters}
%
\begin{Description}\relax
\code{bound.parameters} is a utility function called during estimation runs. It forces the parameter
estimates to remain within the boundaries defined in the .csv file of initial estimates.
\end{Description}
%
\begin{Usage}
\begin{verbatim}
  bound.parameters(x = NULL,
                   lb = NULL,
                   ub = NULL)
\end{verbatim}
\end{Usage}
%
\begin{Arguments}
\begin{ldescription}
\item[\code{x}] A vector of \emph{p} parameter estimates.
\item[\code{lb}] A vector of \emph{p} lower boundaries.
\item[\code{ub}] A vector of \emph{p} upper boundaries.
\end{ldescription}
\end{Arguments}
%
\begin{Value}
Returns a vector of \emph{p} values. The ith element of the returned vector is: \begin{itemize}

\item x[i]  if lb[i] < x[i] < ub[i]
\item lb[i] if x[i] <= lb[i]
\item ub[i] if ub[i] <= x[i]

\end{itemize}

\end{Value}
%
\begin{Author}\relax
Sebastien Bihorel (\email{sb.pmlab@gmail.com})
\end{Author}
%
\begin{Examples}
\begin{ExampleCode}

bound.parameters(seq(1:5), lb=rep(3,5), ub=rep(4,5))

# The following call should return an error message
bound.parameters(1, lb=rep(3,5), ub=rep(4,5))

\end{ExampleCode}
\end{Examples}
