\inputencoding{utf8}
\HeaderA{estimation.plot}{Create Summary Estimation Plots}{estimation.plot}
\keyword{method}{estimation.plot}
%
\begin{Description}\relax
\code{estimation.plot} is a secondary function called at the end of the
estimation runs. It generates plots from the iteration log file and the
prediction \& residual file. Those plots are: a figure summarizing the changes
in the objective function and the estimated parameter values as a function of
the iteration plus, for each subject and sub-problem (i.e. treatment), a 
figure overlaying model predictions and observed data, and another figure 
showing 4 goodness-of-fit plots (predictions vs observations, weighted 
residuals vs time, weighted residuals vs observations, weighted residuals vs 
predictions). See \code{vignette('scaRabee',package='scaRabee')} for more 
details. \code{estimation.plot} is typically not called directly by users.
\end{Description}
%
\begin{Usage}
\begin{verbatim}
  estimation.plot(problem = NULL,
                  Fit = NULL,
                  files = NULL)
\end{verbatim}
\end{Usage}
%
\begin{Arguments}
\begin{ldescription}
\item[\code{problem}] A list containing the following levels:\begin{description}

\item[data] A list which content depends on the scope of the analysis. If 
the analysis was run at the level of the subject, \code{data} contains as 
many levels as the number of subjects in the dataset, plus the \code{ids}
level containing the vector of identification numbers of all subjects 
included in the analysis population. If the analysis was run at the level 
of the population, \code{data} contains only one level of data and 
\code{ids} is set to 1.

Each subject-specific level contains as many levels as there are treatment
levels for this subject, plus the \code{trts} level listing all treatments
for this subject, and the \code{id} level giving the identification number
of the subject. 

Each treatment-specific levels is a list containing the following levels: 
\begin{description}

\item[cov] mij x 3 data.frame containing the times of observations of the
dependent variables (extracted from the TIME variable), the indicators
of the type of dependent variables (extracted from the CMT variable),
and the actual dependent variable observations (extracted from the 
DV variable) for this particular treatment and this particular 
subject.
\item[cov] mij x c data.frame containing the times of observations of 
the dependent variables (extracted from the TIME variable) and all the
covariates identified for this particular treatment and this 
particular subject.
\item[bolus] bij x 4 data.frame providing the instantaneous inputs 
for a treatment and individual.
\item[infusion] fij x (4+c) data.frame providing the zero-order inputs for
a treatment and individual.
\item[trt] the particular treatment identifier.
\end{description}


\item[method] A character string, indicating the scale of the analysis. Should
be 'population' or 'subject'.
\item[init] A data.frame of parameter data with the following columns:
'names', 'type', 'value', 'isfix', 'lb', and 'ub'.
\item[debugmode] Logical indicator of debugging mode.
\item[modfun] Model function.
\item[hbsize] An integer scalar providing the size of the history buffer for
the solver of delay differential equations. This variable is ignored if
the model file does not contain a \$DDE tag.

\end{description}


\item[\code{Fit}] A list containing the following elements:\begin{description}

\item[estimations] The vector of final parameter estimates.
\item[fval] The minimal value of the objective function.
\item[cov] The matrix of covariance for the parameter estimates.
\item[orderedestimations] A data.frame with the same structure as
\code{problem\$init} but only containing the sorted estimated estimates.
The sorting is performed by \code{order.param.list}.
\item[cor] The upper triangle of the correlation matrix for the parameter
estimates.
\item[cv] The coefficients of variations for the parameter estimates.
\item[ci] The confidence interval for the parameter estimates.
\item[AIC] The Akaike Information Criterion.
\item[sec] A list of data related to the secondary parameters, containing
the following elements:\begin{description}

\item[estimates] The vector of secondary parameter estimates calculated
using the initial estimates of the primary model parameters.
\item[names] The vector of names of the secondary parameter estimates.
\item[pder] The matrix of partial derivatives for the secondary
parameter estimates.
\item[cov] The matrix of covariance for the secondary parameter
estimates.
\item[cv] The coefficients of variations for the secondary parameter
estimates.
\item[ci] The confidence interval for the secondary parameter
estimates.

\end{description}


\item[orderedfixed] A data.frame with the same structure as
\code{problem\$init} but only containing the sorted fixed estimates.
The sorting is performed by \code{order.param.list}.
\item[orderedinitial] A data.frame with the same content as
\code{problem\$init} but sorted by \code{order.param.list}.

\end{description}


\item[\code{files}] A list of input used for the analysis. The following elements are
expected and none of them could be null: \begin{description}

\item[data] A .csv file located in the working directory, which contains
the dosing information, the observations of the dependent variable(s)
to be modeled, and possibly covariate information. The expected format 
of this file is described in details in \code{vignette('scaRabee',
        package='scaRabee')}.
\item[param] A .csv file located in the working directory, which contains
the initial guess(es) for the model parameter(s) to be optimized or used
for model simulation. The expected format of this file is described in
details in \code{vignette('scaRabee',package='scaRabee')}.
\item[model] A text file located in the working directory, which defines 
the model. Models specified with explicit, ordinary or delay 
differential equations are expected to respect a certain syntax and 
organization detailed in \code{vignette('scaRabee',package='scaRabee')}.
\item[iter] A .csv file reporting the values of the objective function
and estimates of model parameters at each iteration.
\item[report] A text file reporting for each individual in the dataset the
final parameter estimates for structural model parameters, residual 
variability and secondary parameters as well as the related statistics 
(coefficients of variation, confidence intervals, covariance and 
correlation matrix).
\item[pred] A .csv file reporting the predictions and calculated residuals
for each individual in the dataset.
\item[est] A .csv file reporting the final parameter estimates for each
individual in the dataset.
\item[sim] A .csv file reporting the simulated model predictions for each 
individual in the dataset. (Not used for estimation runs).

\end{description}


\end{ldescription}
\end{Arguments}
%
\begin{Author}\relax
Sebastien Bihorel (\email{sb.pmlab@gmail.com})
\end{Author}
