\inputencoding{utf8}
\HeaderA{scarabee.clean}{Cleaning of the Run Directory}{scarabee.clean}
\keyword{method}{scarabee.clean}
%
\begin{Description}\relax
\code{scarabee.clean} is a secondary function called at each \pkg{scaRabee} 
run. It cleans the run directory from unwanted files. 
\end{Description}
%
\begin{Usage}
\begin{verbatim}
  scarabee.clean(files = NULL,
                 analysis = NULL)
\end{verbatim}
\end{Usage}
%
\begin{Arguments}
\begin{ldescription}
\item[\code{files}] A list of input used for the analysis. The following elements are
expected and none of them could be null: \begin{description}

\item[data] A .csv file located in the working directory, which contains
the observations of the dependent variable(s) to be modeled. The
expected format of this file is described in details in
\code{vignette('scaRabee',package='scaRabee')}.
\item[param] A .csv file located in the working directory, which contains
the initial guess(es) for the model parameter(s) to be optimized or used
for model simulation. The expected format of this file is described in
details in \code{vignette('scaRabee',package='scaRabee')}.
\item[dose] A .csv file located in the working directory, which contains
the dosing information. The expected format of this file is described in
details in \code{vignette('scaRabee',package='scaRabee')}.
\item[cov] A .csv file located in the working directory, which contains
the values of one or more covariates that may or may or be used within
the model. The expected format of this file is described in details in
\code{vignette('scaRabee',package='scaRabee')}.
\item[model] A .R file located in the 'model.definition'
sub-directory in the working directory, which defines the model. Models
specified with explicit, ordinary or delayed differential equations
should be preferentially defined using the provided templates. More
details about the expected structure of this file is provided in
\code{vignette('scaRabee',package='scaRabee')}, in case the user would
want to develop her/his own template.
\item[var] A .R file located in the 'model.definition' sub-directory
in the working directory, which defines the model of residual
variability. More details about the expected content of this file is
provided in \code{vignette('scaRabee',package='scaRabee')}.
\item[sec] A .R file located in the 'model.definition' sub-directory
in the working directory, which defines the method of computation of
secondary parameters (derived from the fixed or estimated model
parameters). More details about the expected content of this file is
provided in \code{vignette('scaRabee',package='scaRabee')}.

\end{description}


\item[\code{analysis}] A character string, defining the 'title' of the run that will
be used to name the output files of the analysis. If
\code{scarabee.analysis} is called within \code{script.R}, a file created
from the provided template , \code{analysis} will be 'script'.
\end{ldescription}
\end{Arguments}
%
\begin{Author}\relax
Sebastien Bihorel (\email{sb.pmlab@gmail.com})
\end{Author}
%
\begin{SeeAlso}\relax
\code{\LinkA{scarabee.analysis}{scarabee.analysis}}
\end{SeeAlso}
