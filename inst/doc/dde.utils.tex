\inputencoding{utf8}
\HeaderA{dde.utils}{Utility Functions for Delay Differential Equation Systems}{dde.utils}
\aliasA{dde.lags}{dde.utils}{dde.lags}
\aliasA{dde.map}{dde.utils}{dde.map}
\aliasA{dde.switch}{dde.utils}{dde.switch}
\aliasA{dde.syst}{dde.utils}{dde.syst}
\aliasA{get.switch.vectors}{dde.utils}{get.switch.vectors}
\keyword{method}{dde.utils}
%
\begin{Description}\relax
This is a collection of utility functions called by \code{dde.model} when 
a model defined by delay differential equations is evaluated. None of these 
functions is typically called directly by users.
\end{Description}
%
\begin{Usage}
\begin{verbatim}
  dde.syst(t = NULL,
           y = NULL,
           dde.parms = NULL)
  dde.lags(parms = NULL,
           derparms = NULL,
           codelags = NULL,
           check = FALSE)
  dde.map(t = NULL,
          y = NULL,
          swID = NULL,
          dde.parms = NULL)
  dde.switch(t = NULL,
             y = NULL,
             dde.parms = NULL)
  get.switch.vectors(dosing = NULL)
\end{verbatim}
\end{Usage}
%
\begin{Arguments}
\begin{ldescription}
\item[\code{t}] A scalar or a vector of numerical time values.
\item[\code{y}] A vector of system state values.
\item[\code{dde.parms}] A list of parameters with the following levels:
\begin{description}

\item[parms] See below
\item[derparms] See below
\item[lags] A vector of delay parameters, typically returned by 
\code{dde.lags}.
\item[codedde] The content of the R code specified within the \$DDE block 
in the model file.
\item[dosing] See below
\item[xdata] A vector of times at which the system is being evaluated.
\item[covdata] A matrix of covariate data extracted from the dataset.
\item[scale] A vector of system scale, typically returned by 
\code{input.scaling}
\item[times] A vector of times need by \code{dde.switch} to determine
the system switches
\item[signal] A vector of signals need by \code{dde.switch} to determine
the system switches
\item[ic] A vector of initial conditions, typically returned by 
\code{init}
\item[check] An indicator whether checks should be performed to validate 
function inputs

\end{description}


\item[\code{parms}] A vector of primary parameters.
\item[\code{derparms}] A list of derived parameters, specified in the \$DERIVED block
of code.
\item[\code{codelags}] The content of the R code specified within the \$LAGS block in
the model file.
\item[\code{check}] An indicator whether checks should be performed to validate 
function inputs.
\item[\code{swID}] A vector of switch ID.
\item[\code{dosing}] A data.frame of dosing information created by \code{make.dosing}
from instantaneous and zero-order inputs into the system.
\end{ldescription}
\end{Arguments}
%
\begin{Details}\relax
\code{dde.syst} is the function which actually evaluates the system of delay
differential equations specified in the \$DDE block.

\code{dde.lags} is the function which evaluates the code specified in the \$LAG 
block and defines the delays at which the system needs to be computed.

\code{dde.map} and \code{dde.switch} are the functions that provide the map 
and switches required by the \code{dde} function. Switches allow the system to
produce discontinuous events (such as an instantaneous bolus). Switches occur
when the function defined in \code{dde.switch} becomes zero due to a change 
from positive to negative values. See vignette of the \pkg{PBSddesolve} 
package for more details.

\code{get.switch.vectors} is the function that creates the data needed for 
\code{dde.switch} to setup the switches.
\end{Details}
%
\begin{Author}\relax
Sebastien Bihorel (\email{sb.pmlab@gmail.com})
\end{Author}
%
\begin{SeeAlso}\relax
\code{\LinkA{dde.model}{dde.model}}, \code{\LinkA{PBSddesolve}{PBSddesolve}}
\code{\LinkA{make.dosing}{make.dosing}}
\end{SeeAlso}
