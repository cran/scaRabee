\inputencoding{utf8}
\HeaderA{fitmle.cov}{Computation of the Covariance Matrix}{fitmle.cov}
\keyword{method}{fitmle.cov}
%
\begin{Description}\relax
\code{fitmle.cov} is a secondary function called during estimation runs. It 
performs multiple tasks after completion of the model optimization by
\code{fitmle}:

1- It computes the matrix of covariance (as described by D'Argenio and 
Schumitzky) by calling \code{get.cov.matrix} and derives some related
statistics: correlation matrix, coefficient of variation of parameter
estimates, confidence intervals and Akaike Information criterion,

2- It estimates secondary parameters and computes the coefficient of variation
of those estimates, as well as the confidence intervals.

\end{Description}
%
\begin{Usage}
\begin{verbatim}
  fitmle.cov(problem = NULL,
             Fit = NULL)
\end{verbatim}
\end{Usage}
%
\begin{Arguments}
\begin{ldescription}
\item[\code{problem}] A list containing the following levels:\begin{description}

\item[data] A list containing the following levels:\begin{description}

\item[xdata] 1 x m matrix of independent variable.
\item[ydata] n x m matrix of observations from model states.
\item[ids] Data.frame of indices for data subsetting (output
from \code{find.id}).

\end{description}

\item[dosing] A list containing the following levels:\begin{description}

\item[history] d x 4 data.frame of dosing history.
\item[ids] data.frame of indices for dosing subsetting
(output from \code{find.id}).

\end{description}

\item[cov] A list containing the following levels:\begin{description}

\item[data] c x t data.frame of covariate history.
\item[ids] Data.frame of indices for cov subsetting (output
from \code{find.id}).

\end{description}

\item[states] Indices of the states to be output by the model.
\item[init] A data.frame of parameter data with the following columns:
'names', 'type', 'value', 'isfix', 'lb', and 'ub'.
\item[debugmode] Logical indicator of debugging mode.
\item[modfun] Model function.
\item[varfun] Variance function; if empty \code{weighting.additive} is
used.
\item[secfun] Secondary parameter function.

\end{description}


\item[\code{Fit}] A list of containing the following levels:\begin{description}

\item[estimations] The vector of final parameter estimates.
\item[fval] The minimal value of the objective function.

\end{description}


\end{ldescription}
\end{Arguments}
%
\begin{Value}
Return a list containing the following elements:\begin{description}

\item[estimations] The vector of final parameter estimates.
\item[fval] The minimal value of the objective function.
\item[cov] The matrix of covariance for the parameter estimates.
\item[orderedestimations] A data.frame with the same structure as
\code{problem\$init} but only containing the sorted estimated estimates.
The sorting is performed by \code{order.param.list}.
\item[cor] The upper triangle of the correlation matrix for the parameter
estimates.
\item[cv] The coefficients of variations for the parameter estimates.
\item[delta] The intervals used for the computation of confidence
intervals.
\item[ci] The confidence interval for the parameter estimates.
\item[AIC] The Akaike Information Criterion.
\item[sec] A list of data related to the secondary parameters, containing
the following elements:\begin{description}

\item[estimates] A vector of secondary parameter estimates.
\item[cov] The matrix of covariance for the secondary parameter
estimates.
\item[cv] The coefficients of variations for the secondary parameter
estimates.
\item[ci] The confidence interval for the secondary parameter
estimates.

\end{description}



\end{description}

\end{Value}
%
\begin{Author}\relax
Sebastien Bihorel (\email{sb.pmlab@gmail.com})

Pawel Wiczling
\end{Author}
%
\begin{References}\relax
D.Z. D'Argenio and A. Schumitzky. ADAPT II User's Guide: Pharmacokinetic/
Pharmacodynamic Systems Analysis Software. Biomedical Simulations Resource,
Los Angeles, 1997.
\end{References}
%
\begin{SeeAlso}\relax
\code{\LinkA{fitmle}{fitmle}}, \code{\LinkA{order.param.list}{order.param.list}}
\end{SeeAlso}
